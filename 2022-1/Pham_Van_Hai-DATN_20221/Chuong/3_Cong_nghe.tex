\documentclass[../DoAn.tex]{subfiles}
\begin{document}
Ở chương 2 đã trình bày tổng quan về các use case của hệ thống, cũng như một số yêu cầu phi chức năng. Để xây dựng ứng dụng theo những yêu cầu đó cần lựa chọn các công nghệ phù hợp. Vì thế ở chương này sẽ trình bày các công nghệ chính và quan trọng nhất được sử dụng trong đồ án.
\section{Xây dựng Restful API với NodeJS, ExpressJS\cite{ExpressJS}}

Bộ đôi NodeJS và ExpressJS hiện nay vẫn là một trong những công nghệ dựng API nổi tiếng nhất bên cạnh những ông lớn như Spring Boot, Django, Laravel. Có rất nhiều công ty lớn trên thế giới sử dụng NodeJS có thể kể đến như Netflix, NASA, Trello, PayPal, LinkedIn, Walmart, Uber, Twitter, Yahoo, eBay, GoDaddy,... Về ExpressJS thì đây là một framework viết trên nền NodeJS.  Tại thời điểm thực hiện DATN này, ExpressJS có lượng tải trung bình của tuần trên npmjs.com là hơn 29 triệu lượt, và trên github.com ExpressJS sở hữu hơn 60 nghìn stars. Sở dĩ ExpressJS được cộng đồng phát triển yêu thích như vậy là do chỉ cần vài dòng mã nguồn là đã có thể tạo ra một API rồi. Chẳng hạn như:
\begin{verbatim}
const express = require('express')
const app = express()
const port = 3000

app.get('/', (req, res) => {
  res.send('Xin chào Việt Nam!')
})

app.listen(port, () => {
  console.log(`Ứng dụng đang chạy trên cổng ${port}`)
})
\end{verbatim}
NodeJS phù hợp với nhiều kiểu ứng dụng và quy mô khác nhau. Với cộng đồng sử dụng đông đảo và kho thư viện khổng lồ viết bằng Javascript càng làm cho NodeJS trở nên mạnh mẽ. Bên cạnh đó, những ứng dụng sử dụng NodeJS cũng có khả năng bảo trì và mở rộng tốt. Trong DATN này, định hướng là sử dụng các công nghệ hỗ trợ tốt nhất cho ngôn ngữ Javascript nên lựa chọn NodeJS cùng với ExpressJS sẽ là hợp lý nhất.

\section{Xây dựng giao diện với ReactJS\cite{ReactJS}}

Để xây dựng giao diện cho ứng dụng em sử dụng thư viện ReactJS, một trong những công cụ mới và được sử dụng rộng rãi bên cạnh những công cụ nổi tiếng khác như Angular, Vue. Hiện nay trên github.com ReactJS sở hữu hơn 202 nghìn stars và trên npmjs.com có trung bình hơn 20 triệu lượt tải hàng tuần, đó là những con số rất ấn tượng. Những điểm nổi bật của ReactJS là phát triển ứng dụng theo hướng component (Chia nhỏ thành các thành phần, tăng cường tái sử dụng), đem đến khái niệm DOM ảo (Virtual DOM) giúp theo dõi và thay đổi chỉ những thành phần có sự thay đổi mà không ảnh hưởng đến các thành phần khác. ReactJS giúp xây dựng các ứng dụng SPA, đem đến trải nghiệm mượt mà, mọi thông tin luôn sẵn sàng khi chuyển đổi giữa các trang chứ không cần phải tải lại cả trang web. So sánh với AngularJS và VueJS thì AngularJS phù hợp hơn cả với những ứng dụng có quy mô lớn, còn VueJS thì phù hợp với những ứng dụng có quy mô từ nhỏ đến trung bình. VueJS cũng là công nghệ mới do đó cộng đồng phát triển cũng chưa thể so sánh với hai đối thủ còn lại. Trong khi đó, ReactJS lại cân bằng nhất khi nó phù hợp với nhiều kiểu và quy mô ứng dụng, có cộng đồng hỗ trợ lớn mạnh và rất nhiều thư viện sinh ra là để dành cho ReactJS. Ngoài ra, nếu người sử dụng có nhiều trải nghiệm với ngôn ngữ Javascript sẽ dễ dàng hơn trong việc làm chủ ReactJS.
\hfill

\section{Xây dựng giao diện với NextJS\cite{NextJS}}
Nếu như VueJS có NuxtJS thì ReactJS có NextJS, NextJS chính là một framework của ReactJS. Trước hết NextJS được dựng trên nền ReactJS nên nó kế thừa những điểm nổi trội của ReactJS. Tuy nhiên, khi sử dụng ReactJS có rất nhiều thứ người dùng cần phải thiết lập thủ công, chẳng hạn như định tuyến, phông chữ, xử lý hình ảnh, xử lý lỗi,... Đó chính là những gì NextJS sẽ hỗ trợ luôn cho người dùng. Một điểm nổi bật nữa của NextJS chính là hỗ trợ SSR giúp trang web có điểm SEO tốt hơn. Nói riêng về DATN này, em muốn có một trang web dành cho khách, ở đây họ có thể tìm hiểu các thông tin về công ty cũng như tìm kiếm việc làm, do đó sử dụng NextJS khả năng SEO tốt nhằm làm trang web nổi bật hơn trên các công cụ tìm kiếm. NextJS cũng sở hữu lượt tải trung bình hàng tuần là gần 4 triệu lượt. Bên cạnh đó NextJS cũng được sử dụng rộng rãi bởi nhiều ông lớn có thể kể đến Netflix, Uber, Starbucks, Twitch,...

\section{Ant Design - Một thư viện xây dựng UI hỗ trợ ReactJS\cite{Antd}}
Ant Design là một trong những thư viện UI hỗ trợ cho ReactJS khá nổi tiếng với trung bình hơn 1 triệu lượt tải hàng tuần trên npmjs.com và hơn 84 nghìn stars trên github.com. Ant Design có bản sắc riêng của mình khi đem đến những "thiết kế" gọn gàng, tinh giản và chú trọng hơn cả vào trải nghiệm khi thao tác. Đó là bởi đội ngữ phát triển của Ant Design đã tìm tòi, khám phá và vận dụng những quy luật khách quan trong tự nhiên về thị giác, xúc giác và cả thính giác của con người để ứng dụng vào trong thiết kế của họ. Khi sử dụng thư viện này, em cũng cần phải tùy chỉnh lại theo thiết kế của riêng mình nhưng vẫn giữ đúng tinh thần của Ant Design bởi lẽ đó cũng là một phần tinh thần của DATN này, mà đó cũng là lý do thư viện UI thì có rất nhiều nhưng Ant Design lại là lựa chọn hàng đầu. Gần đây nhất Ant Design đã ra phiên bản thứ 5, đem đến nhiều tính năng mới rất thú vị và hoàn toàn có thể ứng dụng vào DATN này.

\section{Xác thực người dùng với Firebase Authentication\cite{FirebaseAuthentication}}
Firebase là một sản phẩm do Google phát triển. Firebase cung cấp rất nhiều tính năng như phân tích, cơ sở dữ liệu thời gian thực, lưu trữ dữ liệu trên mây,... Và phải kể đến là tính năng xác thực người dùng. Một ứng dụng cần xác thực người dùng thì cách thông thường sẽ là đăng nhập bằng tài khoải (username/email và password). Tuy nhiên, nếu có thể sử dụng một phương pháp nào đó vẫn xác thực được người dùng nhưng tiết kiệm được thời gian và công sức hơn thì chắc chắn trải nghiệm của người dùng sẽ được nâng cao. Đó là lý do trong DATN này sử dụng Firebase Authentication xác thực bằng tài khoản Google như là một phương pháp xác thực người dùng thay thế cho việc sử dụng tài khoản thông thường đã nêu ở trên. Giờ đây người dùng chỉ cần liên kết tài khoản của mình với một tài khoản Google thì ở những lần đăng nhập sau đó chỉ cần thao tác bằng việc bấm chuột mà thôi. 

\section{Giao tiếp với cơ sở dữ liệu dùng Prisma\cite{Prisma}}
Thông thường để giao tiếp với cơ sở dữ liệu thì ta có thể dùng các câu truy vấn viết bằng ngôn ngữ SQL. Nhược điểm của việc truy vấn bằng cách này chính là dễ xảy ra lỗi đặc biệt trong các tình huống phức tạp cần một câu SQL quá dài dòng. Vì thế các ứng dụng ORM đã ra đời nhằm giải quyết rất nhiều nhược điểm của cách dùng SQL thông thường. ORM giống như một lớp trừu tượng giữa phần cơ sở dữ liệu (Database) và phần điều khiển (Controller) vậy. Trong DATN này sử dụng Prisma chính là một ứng dụng ORM. Người sử dụng sau khi khai báo đường dẫn đến cơ sở dữ liệu và tạo ra các model đại diện cho các bảng trong cơ sở dữ liệu thì có thể thao tác với cơ sở dữ liệu thông qua những phương thức mà Prisma cung cấp. Bên cạnh đó, Prisma cũng như các ORM khác giúp cho ứng dụng an toàn hơn trước các mối nguy hiểm như là SQL Injection.

\section{Triển khai giao diện sử dụng Vercel\cite{Vercel}}
Để deploy một ứng dụng trên internet thì có rất nhiều nền tảng hỗ trợ, bao gồm cả những nền tảng tính phí và miễn phí. Vercel là một trong những nền tảng hỗ trợ triển khai ứng dụng miễn phí, cung cấp tên miền miễn phí và thậm chí cho phép tùy chỉnh tên miền. Để triển khai ứng dụng với Vercel thì một trong những cách đơn giản nhất là để mã nguồn trên một công cụ quản lý mã nguồn ví dụ như Github rồi kết nối với Vercel. Chỉ cần cấu hình những thông tin cơ bản của ứng dụng, Vercel sẽ triển khai ứng dụng lên internet một cách tự động. Ngoài ra, mỗi khi mã nguồn trên Github có sự thay đổi, Vercel sẽ tự động triển khai một phiên bản mới của ứng dụng và điều này là có thể tùy chỉnh được. Có một điểm đặc biệt đó là NextJS được phát triển bởi chính Vercel, cho nên triển khai ứng dụng NextJS lên Vercel sẽ đem lại sự tối ưu nhất định mà không phải nền tảng nào cũng mang lại được.

\section{Triển khai server sử dụng Digital Ocean App Platform\cite{DigitalOcean}}
Digital Ocean là một nhà cung cấp cơ sở hạ tầng trên mây, chuyên cung cấp dịch vụ máy ảo VPS Linux với cấu hình cao và giá cả phải chăng. Cũng giống như Vercel, để triển khai một ứng dụng lên nền tảng này, người dùng có thể kết nối tới một project lưu trữ trên kho lưu trữ mã nguồn (chẳng hạn như Github) và sử dụng Digital Ocean App Platform, thực hiện cấu hình các thông tin cơ bản của dự án và sau đó Digital Ocean App Platform sẽ tự động triển khai ứng dụng theo những cấu hình đó. Mỗi khi có thay đổi mã nguồn trên Git, một bản mới của ứng dụng sẽ được triển khai lại tự động. Digital Ocean App Platform cung cấp giao diện quản lý chuyên nghiệp, giúp người dùng theo dõi được trạng thái của máy chủ, trạng thái của ứng dụng, thay đổi các cài đặt của ứng dụng nếu cần thiết.
\end{document}