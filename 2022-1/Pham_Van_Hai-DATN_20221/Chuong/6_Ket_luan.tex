\documentclass[../DoAn.tex]{subfiles}
\begin{document}
\section{Kết luận}
Từ việc lên ý tưởng đến triển khai vào thực tế, DATN này đã hoàn thành trong đúng thời hạn cho phép và đạt được các yêu cầu đã đề ra. So với các hệ thống khác, về mặt chức năng hệ thống của em còn khá khiêm tốn. Tuy nhiên, hệ thống vẫn có thể hỗ trợ tốt cho bộ phận nhân sự của các công ty. Phần tiếp theo em xin tổng kết lại những điểm đã và chưa làm được, những kiến thức, kỹ năng mới cũng như những bài học kinh nghiệm trong suốt quá trình thực hiện đồ án này.

\subsection{Những gì đã làm được}
Đồ án được thực hiện theo một trình tự rõ ràng. Trước hết em khảo sát các hệ thống hiện có liên quan đến chủ đề của đồ án và khảo sát nhu cầu của một số công ty, từ đó xác định được những gì nên làm, những gì phải có và cân nhắc khối lượng công việc cùng với lượng thời gian cho phép để đảm bảo chất lượng sản phẩm. Tiếp đó, em thực hiện phân tích yêu cầu, phân tích thiết kế hệ thống và thiết kế cơ sở dữ liệu và lựa chọn MySQL là hệ quản trị cơ sở dữ liệu. Tiếp đến em tìm hiểu công nghệ và lựa chọn ReactJS và NextJS để xây dựng phần giao diện, lựa chọn NodeJS và ExpressJS để xây dựng Restful API, cùng nhiều công cụ khác hỗ trợ cho việc phát triển ứng dụng như Novu để hỗ trợ tính năng thông báo, Chatwood hỗ trợ tính năng chat, Cloudinary hỗ trợ lưu trữ file, Mailtrap hỗ trợ giả lập một mail server, Firebase Authentication hỗ trợ đăng nhập bằng tài khoản Google. Sau khi đã có đầy đủ các nguồn lực, em tiến hành xây dựng mã nguồn, kiểm thử hệ thống, triển khai lên internet, khảo sát đánh giá người dùng và định hướng phát triển trong tương lai. Đa số những người dùng thử hệ thống đều đánh giá hệ thống tương đối dễ sử dụng, đáp ứng được phần nào nhu cầu quản lý nhân lực ở các công ty quy mô vừa và nhỏ. Tính năng làm bài kiểm tra online cho ứng viên là tính năng em tâm đắc nhất, em tin rằng tính năng này sẽ giúp ích cho quy trình tuyển dụng ở nhiều công ty. Những người dùng thử hệ thống cũng đều hứng thú với tính năng này.

\subsection{Những gì chưa làm được}
Về dự định ban đầu của em trước khi cân nhắc khối lượng công việc và lượng thời gian cho phép, em mong muốn đồ án này có thể phát triển nhiều tính năng hơn hiện tại. Thứ nhất, hệ thống mới chỉ có tính năng ghi lại thời gian làm việc của nhân viên, là một phần dữ liệu để tính lương mà chưa có tính năng tính lương. Thứ hai, hệ thống chưa hỗ trợ đăng tin tuyển dụng lên các trang web khác mà chỉ có đăng tin tuyển dụng lên chính trang web của công ty. Thứ ba, hệ thống chưa có nhiều tính năng hỗ trợ quá trình tuyển dụng, chẳng hạn tính năng phỏng vấn online, tính năng theo dõi thời gian (time tracking). Phần thống kê báo cáo cũng mới dùng lại ở mức độ thống kê dữ liệu tĩnh, chưa có các phân tích dự đoán (chẳng hạn như tích hợp trí tuệ nhân tạo).

\subsection{Những kiến thức, kỹ năng mới}
Thông qua quá trình làm đồ án này, bản thân em học thêm được rất nhiều kiến thức và kỹ năng mới. Trước hết, em thực sự thấy được tầm quan trọng của việc có một quy trình phát triển phần mềm rõ ràng, khoa học. Ngoài ra, trong ứng dụng này em sử dụng nhiều công cụ của các bên thứ ba, nên em cũng học được cách sử dụng các công cụ này và mường tượng được các dự án trong thực tế khi liên kết với các bên thứ ba cũng sẽ thực hiện theo những cách tương tụ như vậy. Bên cạnh đó, em học được cách triển khai ứng dụng trên internet, vốn là công việc em chưa từng làm trước đó trong suốt thời gian học đại học. Cuối cùng là những kỹ năng mềm, em đã học được kỹ năng khảo sát và phân tích thiết kế, phát triển kỹ năng giải quyết vấn đề, kỹ năng tự học, kỹ năng lập kế hoạch và kỹ năng viết luận. Tất cả những kiến thức và kỹ năng trên đều là những hành trang vô cùng quan trọng giúp em tự tin và vững vàng trong chặng đường phía trước.

\section{Hướng phát triển}
Trước hết cần nâng cấp các tính năng hiện có trong hệ thống. Chẳng hạn chức năng tạo bài kiểm tra, có thể thêm một số loại câu hỏi khác, hoặc thêm kiểu tạo bài kiểm tra khác (ví dụ, vừa tạo bài kiểm tra vừa tạo câu hỏi dùng riêng cho bài kiểm tra đó). Phần thống kê cần thống kê nhiều thông tin hơn và có phần phân tích các dữ liệu ấy. Về phần tìm kiếm sẽ cải thiện chức năng tìm kiếm dữ liệu là text (chẳng hạn dùng Elastic Search) vì tìm kiếm thông thường với những ngôn ngữ có dấu như tiếng Việt có thể không đem lại kết quả chính xác.

Trong tương lai, hệ thống có thể phát triển thêm nhiều tính năng hữu ích như: Tính năng theo dõi thời gian, tính năng lên kế hoạch kết hợp các công cụ tạo lịch online, tính năng phỏng vấn online, tính năng tính lương. Cũng có thể xem xét nâng cấp tính năng chat (chat tự động sử dụng các bots). Ngoài ra có thể xây dựng thêm những tính năng liên quan đến quản lý đãi ngộ, quản lý lương thưởng, quản lý hợp đồng online.

Trên đây là toàn bộ nội dung của đồ án tốt nghiệp với chủ đề \textbf{Hệ thống hỗ trợ tuyển dụng và quản lý nhân sự cho công ty}. Với kiến thức, kỹ năng còn hạn chế và thời gian thực hiện đồ án có hạn nên trong quá trình làm đồ án này, em không thể tránh khỏi những thiếu sót. Em rất mong nhận được sự chỉ bảo, đóng góp của các thầy/cô để đồ án của em ngày càng hoàn thiện hơn. Em xin chân thành cảm ơn.
\end{document}